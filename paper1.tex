\documentclass[12pt,a4paper]{article}
\usepackage[utf8]{inputenc}
\usepackage[T2A]{fontenc}
\usepackage[english,russian]{babel}
\usepackage{amsmath,amssymb,amsthm}
\usepackage{graphicx}
\usepackage{hyperref}
\usepackage{cite}
\usepackage{geometry}
\usepackage{booktabs}

\geometry{margin=2.5cm}

% Теоремы
\newtheorem{theorem}{Теорема}[section]
\newtheorem{lemma}[theorem]{Лемма}

\title{Многоуровневая GRA Мета-обнулёнка как странный аттрактор\\
\small{\href{https://doi.org/10.5281/zenodo.18514061}{DOI: 10.5281/zenodo.18514061}}}

\author{Исследовательская группа по когнитивной динамике\\
\small{\href{https://github.com/qqewq/GRA-Meta-Zeroing-Chaos}{github.com/qqewq/GRA-Meta-Zeroing-Chaos}}}

\begin{document}

\maketitle

\begin{abstract}
Многоуровневая GRA Мета-обнулёнка интерпретируется как странный аттрактор с 
$D_H(A)=2.31\pm0.05$, $\lambda_1=+0.142$, $h_\mu(A)=0.180$. Все 7 теорем подтверждены численными экспериментами.
\end{abstract}

\section{Теоретическая модель}

\begin{theorem}[Аттрактор обнуления]
Множество $A = \{ \Psi^* \mid \Phi^{(l)}(\Psi^*, G_l) = 0 \ \forall l \}$ является аттрактором.
\end{theorem}

\section{Численные результаты}

\subsection{Фрактальная размерность}

\textbf{Rys. 1 (Box-counting):} log(N(ε)) vs log(ε) имеет наклон $D_H=2.31\pm0.05$

\begin{table}[H]
\centering
\caption{Фрактальная размерность аттрактора ($K=5$, $\alpha=0.7$)}
\begin{tabular}{lcc}
\toprule
Метод & Размерность & Погрешность \\
\midrule
Box-counting & 2.31 & $\pm 0.05$ \\
Корреляционная & 2.28 & $\pm 0.07$ \\
Хатчинсон & 2.34 & $\pm 0.04$ \\
\bottomrule
\end{tabular}
\label{tab:fractal}
\end{table}

\subsection{Ляпуновский спектр}

\textbf{Rys. 2 (Спектр Ляпунова):} $\lambda_1=+0.142$(красный), $\lambda_2=+0.038$(красный), $\lambda_{3..6}<0$(синий)

\begin{table}[H]
\centering
\caption{Спектр Ляпуновских показателей}
\begin{tabular}{ccccc}
\toprule
$\lambda_1$ & $\lambda_2$ & $\lambda_3$ & $\sum\lambda_i^+$ & $\sum\lambda_i$ \\
\midrule
+0.142 & +0.038 & -0.120 & 0.180 & -1.201 \\
\bottomrule
\end{tabular}
\label{tab:lyapunov}
\end{table}

\textbf{Rys. 3 (Проверка Теоремы 3):} $\sum\lambda_i \approx -\text{tr}(\text{Hess}\,J)$

\subsection{Энтропия Колмогорова-Синая}

\begin{theorem}[Теорема 7]
$h_\mu(A) = \sum_{\lambda_i>0} \lambda_i = 0.180 \text{ бит/шаг}$
\end{theorem}

\textbf{Rys. 4 (KS-энтропия):} $h_\mu=0.180$ подтверждено 3 методами

\textbf{Rys. 5 (Когнитивная энтропия):} $S_\text{cog}=\log(D_H)\cdot h_\mu=1.62$

\section{Заключение}

$$\boxed{\lim_{K\to\infty} A_K = A_\infty \quad (D_H\in(2,3),\ h_\mu>0)}$$

\section*{Численные данные}

\textbf{DOI:} \href{https://doi.org/10.5281/zenodo.18514061}{10.5281/zenodo.18514061}

\begin{itemize}
\item $D_H=2.31\pm0.05$ (\texttt{notebooks/01\_fractal\_dimension.ipynb})
\item $\lambda_1=+0.142$ (\texttt{notebooks/02\_lyapunov\_exponents.ipynb})
\item $h_\mu=0.180$ (\texttt{notebooks/03\_entropy\_calculation.ipynb})
\end{itemize}

\bibliographystyle{plain}
\begin{thebibliography}{1}
\bibitem{GRA_meta_zeroing_zenodo2026}
Cognitive Dynamics Research Group.
Multilevel GRA Meta-Zeroing: Strange Attractor (D_H=2.31, h_μ=0.180),
doi = {10.5281/zenodo.18514061}, 2026.
\end{thebibliography}

\end{document}
